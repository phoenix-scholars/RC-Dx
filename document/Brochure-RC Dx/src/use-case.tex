% %%%%%%%%%%%%%%%%%%%%%%%%%%%%%%%%%%%%%%%%%%%%%%%%%%%%%%%%%%%%%%%%%%%%%%%%%%%%%%%%%
% حق نشر 1392-1402 دانش پژوهان ققنوس
% حقوق این اثر محفوظ است.
% 
% استفاده مجدد از متن و یا نتایج این اثر در هر شکل غیر قانونی است مگر اینکه متن حق
% نشر بالا در ابتدای تمامی مستندهای و یا برنامه‌های به دست آمده از این اثر
% بازنویسی شود. این کار باید برای تمامی مستندها، متنهای تبلیغاتی برنامه‌های
% کاربردی و سایر مواردی که از این اثر به دست می‌آید مندرج شده و در قسمت تقدیر از
% صاحب این اثر نام برده شود.
% 
% نام گروه دانش پژوهان ققنوس ممکن است در محصولات به در آمده شده از این اثر درج
% نشود که در این حالت با مطالبی که در بالا اورده شده در تضاد نیست. برای اطلاع
% بیشتر در مورد حق نشر آدرس زیر مراجعه کنید:
% 
% http://dpq.co.ir/licence
% %%%%%%%%%%%%%%%%%%%%%%%%%%%%%%%%%%%%%%%%%%%%%%%%%%%%%%%%%%%%%%%%%%%%%%%%%%%%%%%%%
% 
% کاربردها و قابلیت‌ها
% 
% در این بخش کاربردها و قابلیت‌های محصول آورده می‌شود. این کاربردها به صورت فهرست
% وار بوده و جنبه‌های رقابتی را مطرح می‌کند. در صورتی که گزینه‌های انتخابی برای
% کاربر وجود دارد در این بخش آورده شود. برای نمونه حافظه پیش فرض ۱ مگا بایت است و
% قابلیت ارتقاع تا ۱۰ مگابایت را دارد.
%
\section{کاربردها}

همانگونه که پیش از این اشاره شد، این دستگاه برای کنترل سیستم‌هایی به کار گرفته
می‌شوند که دسترسی به آنها دشوار است.
از جمله کاربردهای این دستگاه می‌توان به موارد زیر اشاره کرد:

\begin{itemize}
  \item درب‌های پارکیگ
  \item درب‌های برقی واحد‌های تجاری
  \item صفحه‌های نمایش پروژکتورها
  \item لوسترها و سیستم‌های نور پردازی
  \item موتورهای آب
  \item سیستم‌های تهویه هوا
  \item درب گاراژ‌ها
\end{itemize}

این موارد تنها تعدادی محدود از کاربردهایی است که می‌توان این دستگاه را به کار
گرفت.
